\section{INPUT DATA FILE}


\subsection{Format of Input File}

All information is read in using the ``Taghandler" routine to
search for a ``tag" and then reading the desired information on the
same line as the tag or the following line(s).  Tags are listed
below with description of the data to be read in, example inputs,
and default value(s).  Tags can be in any order and the
Taghandler" routine always finds only the first occurrence of a
tag. Tags listed under section {\bf Tight binding}, {\bf Plane
wave}, and {\bf Plane wave density} are specific to their
respective methods. TB specific tags are ignored by PW and PWD, PW
specific tags are ignored by PWD and TB, etc.

The tags are chosen to coordinate with those used by the SIESTA
code,
http://www.uam.es/departamentos/ciencias/fismateriac/siesta/.

Except were noted otherwise, all input and output parameters are
in atomic units: energy in Hartrees and length in Bohr.

%%%%%%%%%%%%%%%%%%%%%%%%%%%%%%%%%%%%%%%%%%%%%%%%%%%%%%%%%%%%%%%%%%%%%%%

\subsection{General System Descriptors}

\begin{description}
\itemsep 10pt
\parsep 0pt

\item[{\bf NumberOfAtoms}] ({\it integer}):
\index{NumberOfAtoms@{\bf NumberOfAtoms}} Number of atoms per
primitive cell in the calculation.  For an isolated system this is
the total number of atoms.

\begin{verbatim}
NumberOfAtoms 2
\end{verbatim}

{\it Default value:} No default. Variable must be supplied.

\item[{\bf NumberOfDimensions}] ({\it integer}):
\index{NumberOfDimensions@{\bf NumberOfDimensions}} This defines the
dimension of the space used the calculation.  Lattice vectors and
basis vectors (atom coordinates) are stored in arrays of dimension
ndim read in after NumberOfDimensions tag. In PW this means the space
dimension, which can be any dimension $1,2,3, \ldots$. In TB this
defines the dimension of the lattive and atomic coordinates; the
orbitals included in the calculation are defined by the model and/or
input file.  (For example, a line of atoms is defined by ndim =1,
whereas a nanotube is defined in a 3-dimensional space with ndim
=1. For a nanotube, the orbitals can be chosen to be the full set of
s, p$_x$, p$_y$, p$_z$, or one can treat only the $\pi$ orbitals, etc.)

\begin{verbatim}
NumberOfDimensions 3
\end{verbatim}

 {\it Default value:} 3

\item[{\bf NumberOfSpecies}] ({\it integer}):
\index{NumberOfSpecies@{\bf NumberOfSpecies}} Number of different
atomic species in the calculation.  Atoms of the same species, but
with different potential parameters are counted as different
species. As of this moment, multiple species have no function in
the PW code.
\begin{verbatim}
NumberOfSpecies 2
\end{verbatim}

{\it Default value:} 1

\item[{\bf ChemicalSpeciesLabel}] ({\it data block}):
\index{ChemicalSpeciesLabel@{\bf ChemicalSpeciesLabel}}
It specifies the different chemical species\index{species} that are
present, assigning them a number for further identification.

\begin{verbatim}
ChemicalSpeciesLabel
1 31 Ga
2 33 As
\end{verbatim}
The first number in the line is the species number (assumed to be
in sequential order), it is followed by the atomic number, and
then by the desired label.

In PW the label must be one of the labels for which there is a
defined potential.  At present, potentials provided in the code
are limited to: Ga, As, and Si (empirical pseudopotentials); H
(Coulomb potential screened with Thomas-Fermi function for r$_s$ =
1.0); and {\tt El} (Empty lattice - free electrons with zero
potential). Additional potentials can be added to module located
in {\tt atomPotentialMod.f90}.

In TB the label merely is a convenient label for the species. For
example one can chose input such as

\begin{verbatim}
ChemicalSpeciesLabel
 1  14   Si
 2  14   Si_surface
\end{verbatim}

{\it Default value:} No default.  Input must be supplied.

\end{description}

%%%%%%%%%%%%%%%%%%%%%%%%%%%%%%%%%%%%%%%%%%%%%%%%%%%%%%%%%%%%%%%%%%%%%%%

\subsection{Lattice and coordinates}

\begin{description}
\itemsep 10pt
\parsep 0pt


\item[{\bf LatticeConstant}] ({\it real}):
\index{LatticeConstant@{\bf LatticeConstant}}
This defines the scale of the lattice vectors (Bohr).
\begin{verbatim}
LatticeConstant 10.6769569
\end{verbatim}
{\it Default value:} 1 Bohr

\item[{\bf LatticeVectors}] ({\it data block}):
\index{LatticeVectors@{\bf LatticeVectors}} The lattices vectors
are read in units of the lattice constant defined above.  There
are NumberOfDimensions basis vectors each with NumberOfDimensions
components.

\begin{verbatim}
LatticeVectors
0.0 0.5 0.5
0.5 0.0 0.5
0.5 0.5 0.0
\end{verbatim}

{\it Default value:} No default.  Variable must be supplied.

\item[{\bf AtomicCoordinatesFormat}] ({\it string}):
\index{AtomicCoordinatesFormat@{\bf AtomicCoordinatesFormat}}
Character string to specify the format of the atomic positions in
input. These can be expressed in two forms:
\begin{itemize}
\item {\tt ScaledCartesian} (atomic positions are given
in Cartesian coordinates, in units of the lattice constant)
\item {\tt ScaledByLatticeVectors} (atomic positions
are given referred to the lattice vectors)
\end{itemize}

{\it Default value:} {\tt ScaledByLatticeVectors}

\item[{\bf AtomicCoordinatesAndAtomicSpecies}] ({\it data block}):
\index{AtomicCoordinatesAndAtomicSpecies@{\bf AtomicCoordinatesAndAtomicSpecies}}
The data is read as follows:
\begin{verbatim}
From i = 1 to totalNumAtoms
     read: atomCoordinates(:,i) speciesOfAtom(i)
\end{verbatim}

{\it Default value:} No default.  Variable must be supplied.

\item[{\bf InputEnergiesInEV}]: \index{InputEnergiesInEV@{\bf
InputEnergiesInEV}} This tag does not store any value. If this tag
is found, the input energies are in eV. If this tag is omitted,
input energies are in Hartrees. This effects {\bf
OrbitsAndEnergies}, {\bf SKMatrices} in TB and {\bf EnergyCutoff}
in PW. Output energies are still in Hartrees.


\end{description}

%%%%%%%%%%%%%%%%%%%%%%%%%%%%%%%%%%%%%%%%%%%%%%%%%%%%%%%%%%%%%%%%%%%%%%%

\subsection{$k$-point sampling and band structure plots}

\begin{description}
\itemsep 10pt
\parsep 0pt

\item[{\bf KPointsScale}] ({\it string}):
\index{KPointsScale@{\bf KPointsScale}}
Specifies the scale of the $k$-vectors given in the {\bf
KPointsAndLabels} below. The available options are:

\begin{itemize}
\item {\tt TwoPi/a} ($k$-point vector coordinates are given in Cartesian
coordinates, in units of $2\pi/a$, where $a$ is the lattice constant)
\item {\tt ReciprocalLatticeVectors} ($k$-point vectors are given in
reciprocal lattice vector coordinates)
\end{itemize}

{\it Default value:} {\tt ReciprocalLatticeVectors}

\item[{\bf NumberOfKPoints}] ({\it data block}):
\index{NumberOfKPoints@{\bf NumberOfKPoints}}
Read in {\bf NumberOfKPoints} from the input file in units determined by
{\bf KPointsScale}.

\begin{verbatim}
NumberOfKPoints
2
0.0  0.0  0.0
0.25 0.25 0.25
\end{verbatim}

Note that this tag is incompatible with {\bf KPointsAndLabels}. For the
moment, this option has no function outside {\bf PWD} code.

{\it Default value:} No default.  Variable must be supplied.

\item[{\bf KPointsAndLabels}] ({\it data block}):
\index{KPointsAndLabels@{\bf KPointsAndLabels}}
Specifies the lines along which band energies are calculated. Here is an
example for FCC lattice using the option {\tt ReciprocalLatticeVectors}:

\begin{verbatim}
KPointsAndLabels
0.0   0.0   0.0      Ga
0.375 0.375 0.75     K
0.5   0.5   0.5      L
0.0   0.0   0.0      Ga
0.0   0.5   0.5      X
0.25  0.625 0.625    U
\end{verbatim}

The last column (mandatory) is a LaTeX label for use in the band
plot. Note that this tag is incompatible with {\bf
NumberOfKPoints}. This option has no useful function outside of
{\bf PW} and {\bf TB} code.

{\it Default value:} No default.  Variable must be supplied.

\item[{\bf NumberOfLines}] ({\it integer}): \index{NumberOfLines@{\bf
NumberOfLines}} Number of directions in the reciprocal lattice along
which the bands are to be plotted. This value cannot exceed the number
of special $k$-points used {\bf KPointsAndLabels} minus one. For
example, if {\bf NumberOfLines} equals 4, the band structure is only
calculated from Ga -- X in the aforementioned FCC lattice. This tag
is required if using {\bf KPointsAndLabels}.

\begin{verbatim}
NumberOfLines 5
\end{verbatim}

 {\it Default
value:} No default.  Variable must be supplied.

\item[{\bf NumberOfDivisions}] ({\it integer}):
\index{NumberOfDivisions@{\bf NumberOfDivisions}}
The number of divisions per line in $k$-space. This tag is required if
using {\bf KPointsAndLabels}.

\begin{verbatim}
NumberOfDivisions 15
\end{verbatim}
NumberOfBands 10
 {\it Default value:} No
default.  Variable must be supplied.

\item[{\bf NumberOfBands}] ({\it integer}):
\index{NumberOfBands@{\bf NumberOfBands}}
The number of bands to plot must be less than the total number of
orbitals ({\it i.e.} basis).

\begin{verbatim}
NumberOfBands 10
\end{verbatim}

 {\it Default value:} No default.  Variable must
be supplied.

\end{description}
%%%%%%%%%%%%%%%%%%%%%%%%%%%%%%%%%%%%%%%%%%%%%%%%%%%%%%%%%%%%%%%%%%%%%%%

\subsection{Tight binding}

\begin{description}
\itemsep 10pt
\parsep 0pt


\item[{\bf TightBindingModelType}] ({\it integer}):
\index{TightBindingModelType@{\bf TightBindingModelType}} Choice
of model:
\begin{itemize}
\item {\tt 0} General Method using rotation matrices that works
for any angular momentum.  Requires input number of channels for
each specie and input of Slater-Koster arrays with F and K
parameters. See below for description.

\item {\tt 1} Harrison universal parameters. See Harrison, "Elec.
Str. and Props. of Solids". Assumes only one channel for each
specie (default value - no need for input of number of channels.
Must read in on-site energies using OrbitsAndEnergies.  No other
parameters are read in; two center integrals are defined solely in
terms of universal parameters and the distances between neighbors.


\item {\tt 10 or greater} Special models defined by the subroutine
{\tt TBParamsSpecial}.   Parameters are read in using the
TightBindingParameters tag. \underline{This is the only routine
that must be
changed to define a new model.} \\
Examples include:
    \begin{itemize}
\item {\tt 10 } Any crystal with one s-state per site and only one
nearest-neighbor 2-center hopping matrix element.  The value is
defined as first entry under TightBindingParameters which applies
to all neighbors below a cutoff radius defined by the second entry
under TightBindingParameters. All other information is in the
structure and the OrbitsAndEnergies. Examples that can be treated
this way include any of the s-bands for elemental crystals in
texts such as Ashcroft and Mermin, Kittel, etc., and $\pi$ bands
of a graphene layer and a buckyball or nanotube in the pure $\pi$
state approximation.
\begin{verbatim}
TightBindingParameters
2
1.0
1.1
\end{verbatim}

\item {\tt 11} Same as above with first and second neighbor
2-center hopping matrix elements. The values are defined under
TightBindingParameters: Entry 1 - n.n. matrix element; Entry 2 -
cutoff distance for n.n.; Entry 3 - n.n.n. matrix element; Entry 4
- cutoff distance for n.n.n.

\begin{verbatim}
TightBindingParameters
4
1.0
1.1
0.3
2.2
\end{verbatim}

\item {\tt 12} A model of an ionic crystal with two atoms per
cell, one with one s-state and one with p-states. The value is
defined under TightBindingParameters: Entry 1 - n.n. s-p sigma
matrix element; Entry 2 - cutoff distance for n.n.  The onsite
energies are defined under OrbitsAndEnergies.

    \end{itemize}
\end{itemize}

\begin{verbatim}
TightBindingModelType  1
\end{verbatim}

{\it Default value:} 0  (General Method - requires input of
Slater-Koster arrays with F and K parameters)

\item[{\bf MaximumDistance}] ({\it real}):
\index{MaximumDistance@{\bf MaximumDistance}} The interaction
cutoff distance in Bohr~units.

\begin{verbatim}
MaximumDistance 5.5
\end{verbatim}

{\it Default value:} No default.  Variable must be supplied.

\item[{\bf NumChannelsPerSpecie}] ({\it data block}):
\index{NumChannelsPerSpecie@{\bf NumChannelsPerSpecie}} The total
number of electron channels for each chemical species. The
distinct channels corresponds to the different principal quantum
numbers.

\begin{verbatim}
NumChannelsPerSpecie
1 2
\end{verbatim}

The first number is the {\bf ChemicalSpeciesLabel} while the second
number is the total number of channels. If the chemical species were
nitrogen for example, the first channel would be the 1s e$^-$ while the
second channel would include the 2s and 2p e$^-$. {\it Note that the
number of channels does not impose any restriction on the number of
available orbitals.}

{\it Default value:} Default = 1.

%RMM Removed LminAndLmax

\item[{\bf OrbitsAndEnergies}] ({\it data block}):
\index{OrbitsAndEnergies@{\bf OrbitsAndEnergies}} The on-site
Slater-Koster parameters (also called "orbital energies") for each
species, channel and specific orbital are specified in units
determined by the switch {\bf InputEnergiesInEV}.

\begin{verbatim}
OrbitsAndEnergies
1 1 4
0  0  -4.2
1  1   1.715
1  2   1.715
1  3   1.715
1 2 1
0  0   6.68
\end{verbatim}

In the example above, the first and sixth line provides the {\it
species number}, {\it channel number}, and {\it number of
orbitals}. The {\it number of orbitals} does refers to the number
orbitals in the basis for that particular specie and channel.  It
does \underline{not} refer to the number of occupied orbitals.

The second through fifth line and the seventh line specify the
{\it angular momentum}, {\it orbital ID}, and {\it orbital
energy}. There will be such a line for each orbital.

The order of the orbitals is defined for each model type:
\begin{description}
\item {\tt 0} For type 0 (General rotation matrix method) the
orbitals are real spherical harmonics.  For each angular momentum
$\ell$, $2\ell + 1$ orbitals must be listed.  The order is given
below. \item {\tt 1} For type 1 (Harrison universal model) the
orbitals are real spherical harmonics are the same as for model 0,
except that only up to $\ell=1$ is supported at present. \item {\tt
>=10} These are specially coded models and it is up to the user to
determine the order and the meaning of the orbitals.
\end{description}

For a given $\ell$, all $2\ell+1$ orbital energies must be
specified. The only allowed {\it orbital ID} for $\ell = 0$ is 0,
for $\ell = 1$ is 1 thru 3, for $\ell = 2$ is 4 thru 8, for $\ell$
is $\ell^2$ thru $\ell^2 + 2\ell$. The orbital ID refers to the
real orbitals constructed from linear combinations of the
spherical harmonics ($Y_{lm}$). The precise order of the orbitals
is defined as done by many authors, e.g., see
\begin{small}
\begin{verbatim}
http://www.cachesoftware.com/mopac/Mopac2002manual/real_spherical_harmonics.html
\end{verbatim}
\end{small}
for the expressions for the orbitals in order for
$\ell$ up to 3. However, other authors may use a different order.
For a particular $\ell$, the orbital ID $\ell^2$ refer to the real
orbital of the form $Y_{\ell 0}$. The next orbital ID, $\ell^2$ +
1, refers to the real orbital of the form $(Y_{\ell-1} -
Y_{\ell+1}^\ast)/\sqrt{2}$. The next next orbital ID, $\ell^2$ +
2, refers to the real orbital of the form $i(Y_{\ell-1} +
Y_{\ell+1}^\ast)/\sqrt{2}$. After the real orbital whose ID equals
$\ell^2$, the orbitals alternate between those of the form
\begin{eqnarray*}
\frac{1}{\sqrt{2}}\left(Y_{l-m} + (-1)^m Y_{lm}^\ast\right)
\end{eqnarray*}
and those of the form
\begin{eqnarray*}
\frac{i}{\sqrt{2}}\left(Y_{l-m} - (-1)^m Y_{lm}^\ast\right)
\end{eqnarray*}
where $m$ is the azimuthal quantum number.

\begin{table}[ht]
\centering
\caption{Orbital ID for $\ell = 0$ thru $\ell = 2$}
\vspace{0.2cm}
\begin{tabular}{|c|c|}\hline
orbital ID & real orbital \\ \hline
0 & $s$   \\
1 & $p_z$ \\
2 & $p_x$ \\
3 & $p_y$ \\
4 & $d_{3z^2 - r^2}$ \\
5 & $d_{xz}$ \\
6 & $d_{yz}$ \\
7 & $d_{x^2-y^2}$ \\
8 & $d_{xy}$  \\ \hline
\end{tabular}
\end{table}

{\it Default value:} No default.  Variable must be supplied.

\item[{\bf numSKparams}] ({\it integer}): \index{numSKparams@{\bf
numSKparams}} (Used only for TightBindingModelType 0 - General
rotation method.) Number of Slater-Koster parameters to be read
in. It is only necessary to specify the non-zero Slater-Koster
values.
\begin{verbatim}
numSKparams 5
\end{verbatim}

{\it Default value:} No default.  Variable must be supplied for
TightBindingModelType 0.

\item[{\bf SKMatrices}] ({\it data block}): \index{SKMatrices@{\bf
SKMatrices}} (Used only for TightBindingModelType 0 - General
rotation method.)
There will be {\bf numSKparams} lines with the following input:\\
$\ell_i$ $\ell_j$ $m$ $n_i$ $n_j$ species$_i$ species$_j$  $F$ $K$

\begin{verbatim}
SKMatrices
0 0 0 1 1 1 1  0.0 -2.075
0 1 0 1 1 1 1  0.0  2.480816372
1 1 0 1 1 1 1  0.0  2.71625
1 1 1 1 1 1 1  0.0 -0.715
1 0 0 1 2 1 1  0.0 -2.327399971
\end{verbatim}

$F$ contains the distance dependent part of the Slater-Koster matrix,
while $K$ contains the distance independent part. The Slater-Koster
matrix is
\begin{eqnarray*}
K \times R_{ij}^F
\end{eqnarray*}
where $R_{ij}$ is the separation between atoms $i$ and $j$.  where
$F$ is a real number and $K$ is units of determined by {\bf
InputEnergiesInEV} times Bohr$^{-F}$.

The Slater-Koster matrix has many symmetries. It is diagonal in $m$ and
since $\pm m$ are equivalent only $|m|$ is needed. The following
symmetries are also imposed,
\begin{eqnarray*}
F(\ell_j,\ell_i,m,n_j,n_i,\mathrm{species}_j,\mathrm{species}_i) & = &
F(\ell_i,\ell_j,m,n_i,n_j,\mathrm{species}_i,\mathrm{species}_j) \\
K(\ell_j,\ell_i,m,n_j,n_i,\mathrm{species}_j,\mathrm{species}_i) & = &
(-1)^{\ell_i + \ell_j}*K(\ell_i,\ell_j,m,n_i,n_j,\mathrm{species}_i,\mathrm{species}_j)
\end{eqnarray*}

{\it Default value:} Any Slater-Koster parameter not specified is
set to zero. Used only for TightBindingModelType 0.

\end{description}

%%%%%%%%%%%%%%%%%%%%%%%%%%%%%%%%%%%%%%%%%%%%%%%%%%%%%%%%%%%%%%%%%%%%%%%

\subsection{Plane wave}


PW carries out calculations of bands using a plane wave basis. The
calculation is done with a fixed potential ({\underline{not}
self-consistent) which may be any form of a spherical potential
$V(r)$ around each atom site.  For realistic calculations the
potential must be chosen to be a reasonable representation of the
true total potential or pseudopotential in the crystal. The
potential is specified by the label {\bf ChemicalSpeciesLabel} in
which must be one of the labels for which there is a defined
potential. At present, potentials provided in the code are limited
to: Ga, As, and Si (empirical pseudopotentials); H (Coulomb
potential screened with Thomas-Fermi function for r$_s$ = 1.0);
and {\tt El} (Empty lattice - free electrons with zero potential).
Additional potentials can be added to module located in {\tt
atomPotentialMod.f90}.

\begin{description}
\itemsep 10pt
\parsep 0pt


 \item[{\bf EnergyCutoff}] ({\it
real}): \index{EnergyCutoff@{\bf EnergyCutoff}} Defines the plane
wave cutoff (in Hartree units unless the tag {\bf InputEnergyInEV}
is present in the input file).

\begin{verbatim}
EnergyCutOff 6.0
\end{verbatim}

{\it Default value:} No default.  Variable must be supplied.

\item[{\bf DiagonalizationSwitch}] ({\it integer}):
\index{DiagonalizationSwitch@{\bf DiagonalizationSwitch}}
The available methods for calculating eigenvalues and eigenvectors are:
\begin{description}
\item {\tt 0} -- Direct matrix inversion
\item {\tt 1} -- Conjugate gradient method
\end{description}
Conjugate gradient is more efficient than direct matrix inversion for
matrices of rank 700 or larger. 

\begin{verbatim}
DiagonalizationSwitch 0
\end{verbatim}

{\it Default value:} 0

\item[{\bf CGIterationPeriod}] ({\it integer}):
\index{CGIterationPeriod@{\bf CGIterationPeriod}}
Occasionally, the initial trial eigenvector is ``bad.'' {\bf
CGIterationPeriod} is the number of retries. Typical values for this tag
are between 1 and 5.
\begin{verbatim}
CGIterationPeriod 1
\end{verbatim}
{\it Default value:} 1

\item[{\bf CGTolerance}] ({\it real}): \index{CGTolerance@{\bf
CGTolerance}} Specifies the tolerance on the energy eigenvalues;
in Hartree unless the tag {\bf InputEnergyInEV} is present in the
input file.
\begin{verbatim}
CGTolerance 1.d-4
\end{verbatim}

{\it Default value:} No default. Variable must be supplied.


\end{description}

%%%%%%%%%%%%%%%%%%%%%%%%%%%%%%%%%%%%%%%%%%%%%%%%%%%%%%%%%%%%%%%%%%%%%%%

\subsection{Plane wave density}

\begin{description}
\itemsep 10pt
\parsep 0pt

\item[{\bf MonkHorstPack}] ({\it data block}):
\index{MonkHorstPack@{\bf MonkHorstPack}}
The number of division per dimension to sample in the first Brillouin
zone.

\begin{verbatim}
MonkHorstPack
2 2 2
\end{verbatim}

{\it Default value:} Gamma point of the cell.

\item[{\bf NumberOfOccupiedBands}] ({\it integer}):
\index{NumberOfOccupiedBands@{\bf NumberOfOccupiedBands}}
$2\times${\bf NumberOfOccupiedBands} must equal the number of
electrons.

{\it Default value:} No default. Variable must be supplied.

\end{description}
