\section{QUICK START}

\subsection{Compilation}
There is a machine \underline{independent} makefile included in the
source root directory. {\tt Make} should always be run from this
directory. The machine \underline{dependent} makefile are located in
directory named {\tt Sys}. Find a suitable makefile for your
architecture, and then copy or link to {\tt arch.make} located in the
source root directory. Note that {\tt arch.make} is included by the main
makefile.

The sources files are organized in three directories: {\tt TB}, {\tt
PW}, and {\tt Common}. Modules which are shared by both {\bf TB} and
{\bf PW} codes are placed in the {\tt Common} directory.

There are three targets to make:
\begin{itemize}
\item {\bf tb} -- the tight binding code
\item {\bf pw} -- the plane wave code
\item {\bf pwDensity} -- the plane wave code that produces the electron
density on a grid
\end{itemize}

Typing {\tt make} plus the target name will create the create the
executable in its respective source directory.

\subsection{Libraries}
{\bf TBPW} is linked against the LAPACK and BLAS libraries. They can
be downloaded from {\tt http://www.netlib.org/}. The appropriate
variables in the machine \underline{dependent} makefile will require
modification.  Because different compilers handle calls to external
functions differently, the following two preprocessor flags are
available for adding underscores to LAPACK (-DAUSLAPCK) and BLAS
(-DAUSBLAS). If you get undefined reference errors in the external
subroutine calls to LAPACK or BLAS, try using these flags.

\subsection{Running the program}
Some example input files along with their corresponding bands structure
plots are provided in the directory {\tt TBPW/TB/Examples} and {\tt
TBPW/PW/Examples}. The tight binding and plane wave codes are
executed follows:
\begin{verbatim}
./pw 

OR

./tb
\end{verbatim}

At the {\tt Input File Name} prompt, type the filename including any
extension. The band structure output file can be easily plotted with
gnuplot by typing
\begin{verbatim}
gnuplot gnuplot.dat
\end{verbatim}
This also creates a band structure plot file {\tt band.ps}

\subsection{Getting help with the code}
To post questions or comments about {\bf TBPW} send an e-mail to
tbpw@mcc.uiuc.edu. You can subscribe to this mailing list by sending
e-mail to tbpw-subscribe@mcc.uiuc.edu.
