\section{INTRODUCTION}

{\bf TBPW} is an electronic structure code primarily intended for
pedagogical purposes. It is written from the ground-up in a
modular style using Fortran 90. This code is composed of three
distinct parts: empirical tight binding ({\bf TB}), empirical
pseudopotential plane wave ({\bf PW}), and features common to all
band structure codes ({\bf Common}).

The main characteristics of these codes are:
\begin{itemize}
\item Readily provides band structure plots

\item {\bf TB} implements the Slater-Koster 2-center formalism.
Options are provided for simple models (Harrison or user
specified) or a completely general approach using a rotation
matrix formalism that allows the use of orbitals with arbitrary
angular momentum ($\ell$).

\item {\bf PW} implements the plane wave method for any local potential or
pseudopotential.  Several choices for potentials are provided and the
user can add others.  The bands can be found using direct
diagonalization or a conjugate gradient method. Additionally, there is a
plane wave density ({\bf PWD}) code which outputs the electron density
on a grid.
\end{itemize}

The code is coordinated with the description given in the book and
associated web site:
\begin{itemize}
\item {\bf Electronic Structure: Basic Theory and Practical
Methods}, by Richard M. Martin, Cambridge University Press,
2004.\\
(See http://books.cambridge.org/0521782856.htm).

\item {\bf ElectronicStructure.org}  is the web site
(http://ElectronicStructure.org) for new material and updates,
including links to {\bf TBPW} and many other codes.
\end{itemize}

The input uses keywords (``tags") and input data chosen to
coordinate with those used by the {\bf SIESTA} code
(http://www.uam.es/departamentos/ciencias/fismateriac/siesta/),
which is a full self-consistent {\it ab initio} pseudopotential local
orbital code.

 Two technical papers
describing the methods implemented in this code are located in
{\tt Doc}.
\begin{itemize}
\item {\tt tight\_binding.pdf} describes tight binding
plus the rotation matrix implementation used to handle orbitals of
arbitrary angular momentum.
\item {\tt conj\_grad.pdf} describes the conjugate gradient method used
in the PW code.
\end{itemize}
